%%
%% Beuth Hochschule für Technik --  
%%
%% Kapitel 4 - 
%%
%%	

\newpage
[Perkowski]
\section{Messtechnische Identifikation des Störverhaltens der Stecke}
Da das Verhalten der Störübertragungsfunktion zu ermitteln ist, kann wie im Kapitel 3 durchgeführt werden. Die Strecke wird erst in den Arbeitspunkt erregt und dann die Störgröße mit einem Schalter an der Hardware zugeschaltet. Wir haben die Störmessung aufgezeichnet \textit{(Dateiname)}. Nachdem wir die gemessene Sprungantwort betrachten, sind wir aufgrund des typischen Verhalten ausgegangen, dass es sich um PT1-System handelt. 
Da mit Hilfe des Tools \textit{sys\_id} die Zeitachse der Störung eingegrenzt werden kann, kann an dieser Stelle die Identifikation der Störstrecke durchgeführt werden. Das Ergebnis kann im folgenden Bild nachvollzogen werden. (Bild)

$ G(s)_{stoer} = \dfrac{2,6424}{1 + 0,16207} $