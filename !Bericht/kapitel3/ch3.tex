%%
%% Beuth Hochschule für Technik --  
%%
%% Kapitel 3 -
%%
%%	


%%%%%%%%%%%%%%%%%%%%%%%%%%
%% bild einfügen:

%\begin{figure}[h]
%	\begin{center}
%		\includegraphics[scale=0.5]{pic.jpg}
%		\caption{bildbeschreibung - titel}
%       \label{für referenzen}
%	\end{center} 
%\end{figure}


%wenn du auf das bild refenzieren willst schreibst du \ref{...} wo ... = der inhalt von \label{•}

% sys_id = sys\_id ...... für latex _ = \_

%namen von dateien und tools bitte immer italic --> \textit{.....}

% mathematische werte zB 0,5V schreibst du so: $0,5V$

\newpage
[Perkowski]
\section{Messtechnische Identifikation des Steuerverhaltens der Strecke}


\subsection{Dynamisches Verhalten}
um dem arbeitspunk

\subsection{Identifikation der Strecke}
Mit Hilfe des Tools sis\_id.m können wir die Strecke identifizieren. Dafür müssen die Vektoren \textit{ySystem} (Sprungantwort), \textit{uerr} (Sprung) und \textit{t} (Zeitachse) im Workspace bekannt sein. Im Tool \textit{sis\_id.m} muss eine Vorauswahl der Übertragungsglieder erfolgen. Da die Sprungantwort am Anfang eine Verzögerung hat, handelt es sich mindestens um ein PT2-Glied. Nach der Berechnung und Ausprobieren von sys\_id ist im Abb. zu sehen, dass am Ende die Strecke aus einem System besteht, das in Reihe PT1, PT2 und PT3 geschaltet ist. Die Parameter der Übertragungsfunktion in V-Normalform können unter im Feld abgelesen werden. Die Übertragungsfunktion kann mit Button "Model speichern" in der Form von Polynom im Workspace abgespeichert werden.


sis\_id bild


formel aus sis\_id: \\

$ G(s) = \dfrac{1,03}{(1 + 0,003365s) * (1 + 0,1864s) * (1 + 2*0,98029s + (0,091032s)^{2}) }$

gerundeter wert für pollkom

$ G(s) =  \dfrac{1,03}{0,000005s^{4} + 0,0017s^{3	} + 0,043s^{2} + 0,368s + 1 } $