%%
%% Beuth Hochschule für Technik --  RCP Abschlussarbeit
%%
%% Hauptdokument
%%
%% 
%%
%%%%%%%%%%%%%%%%%%%%%%%%%%%%%%%%%%%%%%%%%%%%%%%%%%%%%%%%%%%%%%%%%%%%%
\documentclass[11pt, a4paper,parskip=half]{article}
%% Übersetzen als Entwurf
%\usepackage[entwurf]{bhtThesis}
%% Übersetzen für die Abgabe
\usepackage[abgabe]{bhtThesis}
\typeout{Deckblatt}

%\usepackage{float} %benutze ich für das richtige setzen der Bilder
\usepackage{blindtext}   %für Blindtext in Kapitel 2
\usepackage{listings}
\usepackage{color}
\lstset{ 
  literate={ö}{{\"o}}1
           {ä}{{\"a}}1
           {ü}{{\"u}}1
           {Ö}{{\"O}}1
           {Ä}{{\"A}}1
           {Ü}{{\"U}}1
           {ß}{{\ss}}1
}


\usepackage{trsym}
\usepackage{bytefield}
\definecolor{light-gray}{gray}{0.90} %Code Hintergrundfarbe

\usepackage{longtable}

%%
%% Pfad zu den Bildern
%%
\graphicspath{
  {pictures/},
  {kapitel1/pictures/},
  {kapitel2/pictures/},
  {kapitel3/pictures/},
  {kapitel4/pictures/},
  {kapitel5/pictures/},
  {kapitel6/pictures/},
  {kapitel7/pictures/},
  {kapitel8/pictures/}
}

% julians einstellungen
\definecolor{darkBHT}{rgb}{0,0.5882,0.5529}
\setlength{\parskip}{6pt} 
\setlength{\parindent}{0pt}

\definecolor{lightgray}{rgb}{.9,.9,.9}
\definecolor{darkgray}{rgb}{.4,.4,.4}
\definecolor{purple}{rgb}{0.65, 0.12, 0.82}

\lstdefinelanguage{JavaScript}{
  keywords={typeof, new, true, false, catch, function, return, null, catch, switch, var, if, in, while, do, else, case, break},
  keywordstyle=\color{blue}\bfseries,
  ndkeywords={class, export, boolean, throw, implements, import, this},
  ndkeywordstyle=\color{darkgray}\bfseries,
  identifierstyle=\color{black},
  sensitive=false,
  comment=[l]{//},
  morecomment=[s]{/*}{*/},
  commentstyle=\color{purple}\ttfamily,
  stringstyle=\color{red}\ttfamily,
  morestring=[b]',
  morestring=[b]"
}

\begin{document}
\pagestyle{fancy}


% \maketitle %Deckblatt


\begin{titlepage}
	\begin{center}
		\Large
		Beuth Hochschule für Technik Berlin
		\textcolor{darkBHT}{\rule{\textwidth}{0.2cm}} \\
		\vspace{2 cm}
		
		
		\begin{figure}[htbp]
			\centering 
			\includegraphics[width=9cm]{BHT-Logo-Basis.pdf}  
		\end{figure}
		
		\vspace{1cm}
		
		\Large
		Fachbereich VI - Technische Informatik - Embedded Systems\\
		Fach Rapid Control Prototyping\\
		SS 2014\\
		\vspace{2 cm}
		
		\Huge
		\textbf{Regelung einer simulierten Druckregelstrecke (II)}
		\vspace{2 cm}
	
		\Large
		Eingereicht am \\
		\today % Aktuelles Datum
		\vspace{0.8cm}
		
		Eingereicht von \\
		\begin{tabular}{ll}
			Matthias Hansert & s791744\\
			Marcus Perkowski & s798936\\
			Marcel Burde & s798984\\
		\end{tabular}

	\end{center}
	\vfill
	\textcolor{darkBHT}{\rule{\textwidth}{0.2cm}}
	\vspace{1 cm}
	\normalsize
	
\end{titlepage}

%
% EOF
%


\pagenumbering{roman}
%%%%%%%%%%%%%%%%%%%%%%%%%%%%%%%%%%%%%%%%%%%%%%%%%%%%%%
%%%%%%%%%%%%%%%%%%%%%%%%%%%%%%%%%%%%%%%%%%%%%%%%%%%%%%

\newpage
[Hansert]\\
\begin{huge}
Aufgabenstellung\\

\end{huge}
Entwerfen und optimieren Sie einen Regelkreis, der den Arbeitspunktdruck für alle möglichen Belastungsfälle (Schalter "ein" und anschließend Schalter "aus") möglichst schnell und ohne Übersteuerung der Stellgröße ausregelt und stationär konstant hält.\\

Berücksichtigen Sie dabei, dass primär Störungen ausgeregelt werden sollen. Eine Führung des Kreises in den Arbeitspunkt erfolgt nur nach Inbetriebnahme der Strecke (ca. 1 mal pro Woche).\\

Nehmen Sie (bis auf das Einstecken der notwendigen Kabelanschlüsse und die Betätigung des Störschalters) keine Änderungen (z.B. Veränderung von Potentiometereinstellungen) am Simulationsgerät vor.\\

Das Simulationserät simuliert einen elektrisch steuerbaren hydraulischen Druckgenerator für den Antrieb einer Arbeitsmaschine. Die Anschlußkonfiguration der Eingangs- und Ausgangssignale ist auf dem folgenden Bild (nächste Seite ) dargestellt:\\

\begin{itemize}
\item Mittels einer Steuerspannung u(t), die einen Aussteuerbereich von -10V bis 10V hat, aber nur im positiven Bereich genutzt werden soll, kann der Druck zwischen 0 und einem Maximalwert verstellt werden.

\item $y_{M} (t)$, die Meßgröße des erzeugten Drucks, kann auf der rechten Seite der Anordnung an einer Buchse in Form einer elektrischen Spannung gemessen werden. Die Meßeinrichtung arbeitet linear und der Verstärkungsfaktor beträgt $V =0,08 \dfrac{V}{Bar}$.\\
Die Meßeinrichtung habe PT1-Verhalten, wobei ihre Zeitkonstante klein gegen die der Strecke ist, so daß sie vernachlässigt werden kann.

\end{itemize}


Der Generator arbeitet mit einem Arbeitspunktdruck von 50 Bar bei einer Grundlast, die anliegt, wenn der „Störschalter“ in der Stellung ohne Beschriftung (also nach unten) steht. Durch eine Schalterbewegung in Richtung „ein“ (nach oben) kann eine maximale Entlastung des Druckgenerators simuliert werden.


\begin{figure}[htbp]
	\begin{center}
		\includegraphics[width=15cm]{elektrischeSchaltung.pdf} 
		\caption{Elektrische simulierte Druckregelstrecke}
     \label{RegelObj}
	\end{center} 
\end{figure}


%%%%%%%%%%%%%%%%%%%%%%%%%%%%%%%%%%%%%%%%%%%%%%%%%%
%%%%%%%%%%%%%%%%%%%%%%%%%%%%%%%%%%%%%%%%%%%%%%%%%%
\newpage
[Hansert]\\
\begin{huge}
Auflistung aller abgegebenen Dateien\\

\end{huge}

\begin{itemize}
\item \textbf{01\_Vorbereitung}
	\begin{itemize}
	\item \textit{reinraus2007b.mdl}
	\item \textit{input\_Mittelwert\_Offset.m}
	\end{itemize}
	
\item \textbf{02\_Statische\_Kennlinie}
	\begin{itemize}
	\item \textit{Berechnung\_Kennlinie.m}
	\item \textit{Statische\_Kennlinie.mdl}
	\item \textit{Scope\_Out.mat}
	\item \textit{ScopeIn.mat}
	\end{itemize}
	
\item \textbf{03\_Steuerverhalten\_Strecke}
	\begin{itemize}
	\item \textit{Statische\_Kennlinie.mdl}
	\item \textit{modell\_sys\_id.mat}
	\item \textit{Scope\_Out.mat}
	\item \textit{sprungantwort.mat}
	\item \textit{UebfktAnalyse.mat}
	\end{itemize}
	
\item \textbf{04\_Stoerverhalten\_Strecke}
	\begin{itemize}
	\item \textit{arbeitspunkt.mdl}
	\item \textit{stoer\_err.mat}
	\item \textit{stoerung.mat}
	\item \textit{StorfktAnalyse.mat}

	\end{itemize}
	
\item \textbf{05\_Simulation\_Strecke}
	\begin{itemize}
	\item !!!!! besprechen !!!!!
	\end{itemize}
	
\item \textbf{06\_Reglerentwurf}
	\begin{itemize}
	\item \textit{...}
	\end{itemize}
	
\item \textbf{07\_Simulation\_Regelkreis}
	\begin{itemize}
	\item \textit{simulation\_regelkreis.mdl}
	\end{itemize}
	
\item \textbf{08\_Regler\_Reale\_Strecke}
	\begin{itemize}
	\item \textit{regler\_strecke.mdl}
	\item \textit{Regler\_und\_reale\_Strecke.mdl}
	\end{itemize}
	
\item \textbf{09\_Sonstiges}
	\begin{itemize}
	\item Vorgehensweise\_zur\_Erstellung\_Belegarbeit
	\item Labor-Übung 11 (Simulierte Druckregelstrecke\_2)
	\end{itemize}
	
\item \textbf{10\_Hilfsprogramme}
	\begin{itemize}
	\item polkomp.m
	\item sys\_id
	\item tf2vn.m
	\end{itemize}

\end{itemize}



%%%%%%%%%%%%%%%%%%%%%%%%%%%%%%%%%%%%%%%%%%%%%%%%%%%%%%
%%%%%%%%%%%%%%%%%%%%%%%%%%%%%%%%%%%%%%%%%%%%%%%%%%%%%%


\newpage
\tableofcontents %Inhaltsverzeichnis


\pagenumbering{arabic}
%%%%%%%%%%%%%%%%%%%%%%%%%%%%%%%%%%%%%%%%%%%%%%%%%%%%%%%%%%%%%%%
%% Die Kapitel der Arbeitw
%%
%% Beuth Hochschule für Technik 
%%
%% Einleitung 1
%%
%%

\newpage
\section{Zu regelndes Objekt kennen lernen}
Das Objekt was wir regeln sollen handelt sich um einen elektrischen steuerbaren hydraulischen Druckgenerator für den Antrieb einer Arbeitsmaschine. Zu beachten ist das es sich um eine simulierte Druckregelstrecke handelt.\\

Bekannt ist:
\begin{itemize}
\item Steuerspannung -10V $\leq u(t) \leq$ 10V, wobei nur der positive Bereich betrachtet wird, weil der Druck von 0 bis einem Maximalwert verstellt wird

\item Die Messgröße $Y_{M}(t)$ wird in Form einer Spannung gemessen

\item Der Verstärkungsfaktor der Messeinrichtung $V_{M}$ beträgt $0,08\frac{V}{Bar}$

\item Bei einer Grundlast liegt der Arbeitspunktdruck bei 50 Bar

\item Die Messeinrichtung hat ein PT1-Verhalten wo die Zeitkonstante vernachlässigt werden kann 
\end{itemize}


Wenn man das Objekt (Abbildung \ref{RegelKreis}) genauer betrachtet, können die Teile eines Standard-Regelkreises erkannt werden:
\begin{itemize}
\item $u(t)$ Regelgröße
\item $y(t)$ Stellgröße
\item Stelleinrichtung und Regelstrecke
\end{itemize}

\begin{figure}[htbp]
	\begin{center}
		\includegraphics[scale=0.4]{regelkreis.pdf}
		\caption{Elektrische simulierte Druckregelstrecke mit Standard-Regelkreis}
       \label{RegelKreis}
	\end{center} 
\end{figure}

\newpage

Um alle Größen zu bestimmen/messen und der Reger zu entwickeln wird MATLAB/Simulink eingesetzt. Dies geschieht mit das Tool Real-Time-Workshop und eine A/D-D/A Wandlerkarte.


\subsection{Offset}
Um das System mit dem Rechner zu verbinden wird eine Wandlerkarte eingesetzt. Die Wandlerkarte ist im PC eingebaut. Um Fälschungen im Ergebnis zu umgehen muss der Eingang- und Ausgangsoffset gemessen werden. Um diese Werte zu messen wurde das \textit{reinraus2007b.mdl} Simulinkmodell benutzt.\\

Für den Eingangsoffset der Wandlerkarte wurde ein Display 



%Für das bestimmen der Offsetwerte haben wir das reinraus2007b.mdl Simulinkmodell benutzt. Bei dem Input wurde ein Display als Offset-Ausgabe benutzt und der entsprechende Gegenoffset über einen Summenpunkt addiert oder subtrahiert. Wichtig hier ist das man die beiden Leitungen die von der Wandlerkarte zur simulierten Druckeregelstrecke gehen miteinander kurz schließt. Für den Output müssen die zwei Leitungen an ein Messgerät angeschlossen werden, dies dann den vorhandenen Offset anzeigt. Im Simulinkmodell wird, wie für den Input, dann der Gegenoffset addiert oder subtrahiert.

%%%%%%%%%%%%%%%%%%%%%%%%%%
%% bild einfügen

%\begin{figure}[h]
%	\begin{center}
%		\includegraphics[scale=0.5]{pic.jpg}
%		\caption{bildbeschreibung - titel}
%       \label{für referenzen}
%	\end{center} 
%\end{figure} %

%%
%% Beuth Hochschule für Technik --  
%%
%% Kapitel 2 - 
%%
%%	
\chapter{Chapter 2}
 %
%%
%% Beuth Hochschule für Technik --  
%%
%% Kapitel 3 -
%%
%%	

\newpage

\section{section33}
 %
%%
%% Beuth Hochschule für Technik --  
%%
%% Kapitel 4 - 
%%
%%	

\newpage

\section{section44} %
%%
%% Beuth Hochschule für Technik --  
%%
%% Kapitel 5 - 
%%
%%	

\newpage

[Perkowski]
\section{Simulation des Steuer- und Störverhaltens der Strecke} \label{Kapitel5}

erläuterung plus bilder simulink und plot %
%%
%% Beuth Hochschule für Technik --  
%%
%% Kapitel 6 - 
%%
%%	

\newpage


\section{Entwurf des Reglers}
 %
%%
%% Beuth Hochschule für Technik --  
%%
%% Kapitel 7 - 
%%
%%	

\newpage


\section{Simulation des Regelkreises mit dem entworfenen Regler}
 %
%%
%% Beuth Hochschule für Technik --  
%%
%% Kapitel 5 - 
%%
%%	

\newpage


\section{Implementierung des Reglers in den realen Regelkreis}

 %


%%%%%%%%%%%%%%%%%%%%%%%%%%%%%%%%%%%%%%%%%%%%%%%%%%%%%%%%%%%%%%%%
%Abbildungsverzeichnis
\newpage
\listoffigures
		
%\newpage
%\begin{appendix}
%  %%
%% Beuth Hochschule für Technik --  Abschlussarbeit
%%
%% Anhang
%%
%%%%%%%%%%%%%%%%%%%%%%%%%%%%%%%%%%%%%%%%%%%%%%%%%%%%%%%%%%%%%%%%%%%%%


\section{Anhang}



%\end{appendix}



%%%%%%%%%%%%%%%%%%%%%%%%%%%%%%%%%%%%%%%%%%%%%%%%%%%%%%%%%%%%%%%
%% Literaturverzeichnis

\clearpage
\newpage
\addcontentsline{toc}{section}{Literatur- und Quellenverzeichnis}
\nocite{*}
\bibliographystyle{IEEEtran}
\bibliography{bhtThesis}

\end{document}
