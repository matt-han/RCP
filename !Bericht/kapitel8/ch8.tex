%%
%% Beuth Hochschule für Technik --  
%%
%% Kapitel 8 - 
%%
%%	

\newpage

[Burde]
\section{Implementierung des Reglers in den realen Regelkreis}

Der kontinuierliche Regler wird nun in den realen Regelkreis implementiert. Dies wird mit Hilfe einer im PC eingebauten A/D Wandlerkarte sowie dem Realtimeworkshop realisiert. Der Regler ist dabei noch immer in einem Modell in Simulink untergebracht. Das Simulink Modell wird in Abbildung \ref{final_modell} dargestellt.

\begin{figure}[h]
	\begin{center}
		\includegraphics[scale=0.9]{modell.jpg}
		\caption{Der Regler im Simulink Modell mit Softwareschnittstellen zur A/D Wandlerkarte}
       \label{final_modell}
	\end{center} 
\end{figure}

\newpage

In der Abbildung \ref{scopes_real} ist nun die Stellgröße und die Regelgröße des realen Regelkreises dargestellt. Die Störung wird nach ca. 6,9 Sekunden eingeschaltet und nach ca. 13 Sekunden wieder ausgeschaltet.

\begin{figure}[h]
	\begin{center}
		\includegraphics[scale=0.9]{scopes.jpg}
		\caption{Messungen der Regelung in der realen Strecke}
       \label{scopes_real}
	\end{center} 
\end{figure}

Aus den Diagrammen aus Abbildung \ref{scopes_real} lassen sich einige positive Schlussfolgerungen ziehen. Der Regler regelt die Störgrößen im realen Regelkreis schnell aus, ohne dabei die Stellgröße auf über $10V$ zu übersteuern und ohne die Überschwingweite $M_{p}$ prägnant zu überschreiten. Zudem gibt es keine bleibende Regelabweichung.